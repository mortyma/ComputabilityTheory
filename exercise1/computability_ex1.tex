   \documentclass [11pt]{article}
   \usepackage{latexsym}
   \usepackage{amssymb}
   \usepackage{url}
   \usepackage{amsmath}
   \usepackage{amsthm}
   \usepackage{paralist}
   \usepackage[normalem]{ulem}

\newcommand{\IF}{\text{\underline{if} }}
\newcommand{\THEN}{\text{ \underline{then} }}
\newcommand{\ELSE}{\text{ \underline{else} }}

\newcommand{\ITEt}[3]{\ensuremath{\IF t_{#1} \THEN #2 \ELSE #3}}
\newcommand{\ITE}[2]{\ensuremath{\IF t \THEN #1 \ELSE #2}}
\newcommand{\ADD}[2]{\ensuremath{A_{#1} \THEN #2}}
\newcommand{\SUB}[2]{\ensuremath{S_{#1} \THEN #2}}
\newcommand{\gR}[1]{\ensuremath{R \THEN #1}}
\newcommand{\gL}[1]{\ensuremath{L \THEN #1}}
\newcommand{\dI}[1]{\ensuremath{d_I \THEN #1}}
\newcommand{\dB}[1]{\ensuremath{d_B \THEN #1}}
  
\newcommand{\PR}{\textbf{PR}}
\newcommand{\R}{\textbf{R}}

\newtheorem{theorem}{Theorem}
\newtheorem{definition}{Definition}
  
\title{Computability Theory: Exercise sheet 1}
\author{Martin Kalany}
\date{\today}

\begin{document}
\maketitle

%%%%%%%%%%%%%%%%%%%%%%%%%%%%%%%%%%%%%%%%%%%%%%%%%%%%%%%%%%%%%%%%%%%%%%%%%%%%%%%%%%%%%%%%%%%%
\bigskip
\noindent
\textbf{Exercise 1:}
For convencience, we define the macros RI, LI as in the lecture:
\begin{compactitem}
\item RI $= (i,  \{
i:\;\gR{i+1},\;
i+1:\;\ITE{i+2}{i}
\})$
\item LI $= (i,  \{
i:\;\gL{i+1},\;
i+1:\;\ITE{i+2}{i}
\})$
\end{compactitem}
We define the program as $P=(1,A)$, where $A = \{$ 
\begin{compactenum}[1:]
\item \gR{2}
\item \ITE{0}{3}
\item \dB{4}
\item RI
\item RI
\item \dI{7}
\item \gR{8}
\item \dI{9}
\item LI
\item LI
\item \gR{2}
\end{compactenum}
$\}$.


%%%%%%%%%%%%%%%%%%%%%%%%%%%%%%%%%%%%%%%%%%%%%%%%%%%%%%%%%%%%%%%%%%%%%%%%%%%%%%%%%%%%%%%%%%%%
\bigskip
\noindent
\textbf{Exercise 2:}
We define the program as $P=(1,A)$, where $A = \{$ 
\begin{compactenum}[1:]
\item \ITEt{1}{4}{2} 
\item \ADD{2}{3}
\item \SUB{1}{1}
\item \ADD{1}{5}
\item \ITE{2}{0}{6}
\item \SUB{2}{7}
\item \ITEt{3}{10}{8}
\item \ADD{3}{9}
\item \SUB{1}{7}
\item \ITEt{3}{5}{11}
\item \ADD{1}{12}
\item \ADD{1}{13}
\item \SUB{3}{10}
\end{compactenum}
$\}$.


%%%%%%%%%%%%%%%%%%%%%%%%%%%%%%%%%%%%%%%%%%%%%%%%%%%%%%%%%%%%%%%%%%%%%%%%%%%%%%%%%%%%%%%%%%%%
\bigskip
\noindent
\textbf{Exercise 3:}
$Func(P): f(x,y) = x*y$

%%%%%%%%%%%%%%%%%%%%%%%%%%%%%%%%%%%%%%%%%%%%%%%%%%%%%%%%%%%%%%%%%%%%%%%%%%%%%%%%%%%%%%%%%%%%
\bigskip
\noindent
\textbf{Exercise 4:}
Let $F \in \{\PR, \R\}$. 
We assume that $f \in F$.
Note that the semantics of the $\Sigma$ operator adhere to the following equation:
\begin{align*}
\Sigma(f)(x,0) &= f(x,0) \\
\Sigma(f)(x, y+1) &= f(x,y+1) + \Sigma(f)(x,y)
\end{align*}
which we can rewrite as
\begin{align*}
\Sigma(f)(x,0) &= C(f; I^1, C^1_0)(x) \\
\Sigma(f)(x, y+1) &= C(plus; C(f; I^3_1, S \circ I^3_2), I^3_3) (x,y, \Sigma(f)(x,y))
\end{align*}
Let $g = C(f; I^1, C^1_0)$ and $h = C(plus; C(f; I^3_1, S \circ I^3_2), I^3_3)$. 
Since $g$ and $h$ only make use of the functions $f$ (which we assume to be a total function) and $plus$ (Example 3.0.4 showed that $plus \in cl_{Pr}(BF\cup S)$) as well as operators in $BF \cup S$, they are total functions.
Thus, we can equivalently write $\Sigma(f) = Pr(g, h)$, i.e, $\Sigma(f) \in cl_{Pr}(BF \cup S)$. 
We conclude that $\Sigma(f) \in \PR$ (immediate by the definition) and $\Sigma(f) \in \R$ (since $\Sigma(f) \in cl_{\{Pr \cup \mu \}}(BF \cup S)$ and total). 

\medskip
\noindent
The proof for the operator $\Pi$ is analogous (just substitute $plus$ by $times$).
\qed


%%%%%%%%%%%%%%%%%%%%%%%%%%%%%%%%%%%%%%%%%%%%%%%%%%%%%%%%%%%%%%%%%%%%%%%%%%%%%%%%%%%%%%%%%%%%
\bigskip
\noindent
\textbf{Exercise 5:}
Let $F \in \{\PR, \R\}$. 
We define $P^a(x,k)$ and $P^e(x,k)$ via the $Pr$ operator:
\begin{align*}
P^a(x, 0) &= P(x,0) \\
P^a(x, y+1) &= P(x,y+1) \land P^a(x,y) \\
\implies P^a &= Pr(C(P; I^1, C^1_1),  C(\land; C(P; I^3_1, S \circ I^3_2), I^3_3) 
\end{align*}
\begin{align*}
P^e(x, 0) &= P(x,0) \\
P^e(x, y+1) &= P(x,y+1) \lor P^e(x,y) \\
\implies P^e &= Pr(C(P; I^1, C^1_1),  C(\lor; C(P; I^3_1, S \circ I^3_2), I^3_3) 
\end{align*}
Since $P(x,y) \in Pred^{k+1}(F)$ and $F$ is closed under $\land$ resp.~$\lor$ (shown in Theorem 3.0.1) and all other functions and operators are basic ones, we conclude that $P^a(x,k),\; P^e(x,k) \in F$.
\qed

%%%%%%%%%%%%%%%%%%%%%%%%%%%%%%%%%%%%%%%%%%%%%%%%%%%%%%%%%%%%%%%%%%%%%%%%%%%%%%%%%%%%%%%%%%%%
\bigskip
\noindent
\textbf{Exercise 6:}
For convencience, we define the function $max: \mathbb{N}^2 \rightarrow \mathbb{N}$ as:
\begin{align*}
max(x,y) = \begin{cases}
           x & \text{ if } x > y \\
           y & \text{ otherwise }
           \end{cases}
\end{align*}
Since $PR$ is closed under $>$ and definition by cases, it is immediate to see that $max$ is primitive recursive.
We define $\text{max}Q$ as
\begin{align*}
(\text{max}Q)(x,0) &= 0 \\
(\text{max}Q)(x,y+1) &= max(Q(x,y+1) * (y+1), (\text{max}Q)(x,y)) \\
\implies (\text{max}Q) &= Pr(C^1_0, C(max; C(times; C(Q; I^3_1, S \circ I^3_2), S \circ I^3_2) , I^3_3))
\end{align*}

Since $Q(x,y) \in Pred^{n+1}(\PR)$ and $\PR$ is closed under $max$ and $times$ (shown in Theorem 3.0.5) and all other functions and operators are basic ones, we conclude that $(\text{max}Q) \in \PR$.
\qed

%%%%%%%%%%%%%%%%%%%%%%%%%%%%%%%%%%%%%%%%%%%%%%%%%%%%%%%%%%%%%%%%%%%%%%%%%%%%%%%%%%%%%%%%%%%%
\bigskip
\noindent
\textbf{Exercise 7:}
We define the function $min: \mathbb{N}^2 \rightarrow \mathbb{N}$ as 
\begin{align*}
min(x,y) = \begin{cases}
           x & \text{ if } x < y \\
           y & \text{ otherwise }
           \end{cases}
\end{align*}
Since $PR$ is closed under $<$ and definition by cases, it is immediate to see that $min$ is primitive recursive.
We define the bounded minimazation operator $\text{min}Q$ analogously to $\text{max}Q$: 
Let $Q \in Pred^{n+1}(\PR)$ and 
\begin{align*}
(\text{min}Q)(x,0) &= 0 \\
(\text{min}Q)(x,y+1) &= min(Q(x,y+1) * (y+1), (\text{min}Q)(x,y)) \\
\implies (\text{min}Q) &= Pr(C^1_0, C(min; C(times; C(Q; I^3_1, S \circ I^3_2), S \circ I^3_2) , I^3_3))
\end{align*}
By the same arguments (just swap $min$ for $max$) as in Exercise 5, we conclude that $(\text{min}Q) \in \PR$. 
Using the definition of the predicate $Prime$ from Example 3.0.12, we define the two-place predicate $pgt$ (prime and greater than) as
\begin{align*}
pgt(x,y) = x < y \land Prime(y) 
\end{align*}
which is trivially in $\PR$.
\begin{theorem}[Bertrand-Tschebyschow]
$\forall n\in \mathbb{N}, n > 1 : \exists p \text{ s.t.~}  n < p \leq 2n \land p \text{ is prime.}$
\end{theorem}

\noindent
Using the well known Theorem of Bertrand-Tschebyschow, we can define the function $nextp: \mathbb{N} \rightarrow \mathbb{N}$
\begin{align*}
nextp(x) &= 
\begin{cases}
  \text{min}\{k \mid x < k \land k \leq 2*x \land Prime(k) \} & \text{if }\exists(x < i \leq 2*x)\; Prime(x)\\
  0 & \text{otherwise}
\end{cases}
\end{align*}
(which computes the next prime greater than $x$) as
\begin{align*}
nextp(x) &= (\text{min}pgt)(x, 2*x)\\
\implies nextp &= C((\text{min}pgt); I^1, C(times; C^1_2, I^1)) \quad .
\end{align*}
Since $nextp$ is a composition of $(\text{min}pgt)$ is a recursive predicate, the identity and constant function and $times \in \PR$, we can cconclude that $nextp \in \PR$.
Finally, we get 
\begin{align*}
p(0) &= 1 \\
p(y+1) &= nextp(p(y)) \\
\implies p &=  Pr(nextp \circ I^2_2)
\end{align*}
which is trivially in $\PR$ (apart from $nextp$, only basic functions and the $Pr$ operators are used).
\qed


\end{document}
