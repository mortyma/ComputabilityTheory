   \documentclass [11pt]{article}
   \usepackage{latexsym}
   \usepackage{amssymb}
   \usepackage{url}
   \usepackage{amsmath}
   \usepackage{amsthm}
   \usepackage{paralist}
   \usepackage[normalem]{ulem}
   \usepackage{enumitem} 
   \usepackage{multicol}

\newtheorem{theorem}{Theorem}
\newtheorem{definition}{Definition}

\newcommand{\E}{\ensuremath{\mathcal{E}}}
\newcommand{\N}{\ensuremath{\mathbb{N}}}
\newcommand{\PR}{\textbf{PR}}
\newcommand{\R}{\textbf{R}}
\renewcommand{\P}{\textbf{P}}
\newcommand{\lra}{\ensuremath{\leftrightarrow}}

  
\title{Computability Theory: Exercise sheet 2}
\author{Martin Kalany}
\date{\today}

\begin{document}
\maketitle

%%%%%%%%%%%%%%%%%%%%%%%%%%%%%%%%%%%%%%%%%%%%%%%%%%%%%%%%%%%%%%%%%%%%%%%%%%%%%%%%%%%%%%%%%%%%
\bigskip
\noindent
\textbf{Exercise 1:} Let $G \in R^1_1$ and $g$ be strictly monotone (increasing) (i.e., $g(/n) < g(n+1)$ for all $n$). Prove that $g(\N)$ is decidable.

\noindent
\textbf{Solution:}

%%%%%%%%%%%%%%%%%%%%%%%%%%%%%%%%%%%%%%%%%%%%%%%%%%%%%%%%%%%%%%%%%%%%%%%%%%%%%%%%%%%%%%%%%%%%
\bigskip
\noindent
\textbf{Exercise 2}  Let \E\ be the set of all recursively enumberable subsets of \N. Prove that 
\begin{enumerate}[label={\alph*)}] 
 \item \E\ is closed under union and intersection.
 \item \E\ is not closed under complement.
\end{enumerate}
\textbf{Solution 2a:}
We have to show that
\begin{enumerate}[label={\alph*)}] 
 \item $A, B \in \E \implies A\cup B \in \E$ 
 \item $A, B \in \E \implies A\cap B \in \E$ 
\end{enumerate}
By Theorem 6.0.2 we get 
\begin{enumerate}[label={\alph*)}] 
 \item $\exists g \in R_1^1 \text{ s.t.\ } g(\N) = A \text{ (i.e., } Range(g) = A)$
 \item $\exists h \in R_1^1 \text{ s.t.\ } h(\N) = B\text{ (i.e., } Range(h) = B)$
\end{enumerate}
We use the definition of recursively enumerable sets (\textbf{Definition 6.0.1}, for $k=1$):

\noindent
\emph{A set $M \subseteq \N$ is called recursively enumerable if there exists an $f \in \textbf{P}$ s.t.\ $D(f) = M$.}

\noindent
and define functions $f_a, f_b$ as:
\begin{multicols}{2}
\noindent
\begin{align*}
& \text{\underline{function} } f_a(x): \\
& \quad \text{\underline{begin}} \\
& \quad \quad i \leftarrow 1; \\
& \quad \quad \text{\underline{while}} (g(i) \neq x \land h(i) \neq x) \\
& \quad \quad \quad \text{ \underline{do} } i \leftarrow i +1; \\
& \quad \quad f_a \leftarrow 0; \\
& \quad \text{\underline{end}}
\end{align*}
\begin{align*}
& \text{\underline{function} } f_b(x): \\
& \quad \text{\underline{begin}} \\
& \quad \quad i \leftarrow 1; \\
& \quad \quad \text{\underline{while}} (g(i) \neq x) \text{ \underline{do} } i \leftarrow i + 1; \\
& \quad \quad j \leftarrow 1; \\
& \quad \quad \text{\underline{while}} (h(j) \neq x) \text{ \underline{do} } j \leftarrow j + 1; \\
& \quad \quad f_b \leftarrow 0; \\
& \quad \text{\underline{end}}
\end{align*}
\end{multicols}
\noindent
Since $g$ and $h$ are recursive, $f_a$ and $f_b$ are clearly partial recursive. 
Note that 
$$
f_a(x) \text{ is defined } \iff \exists i \in \N: g(i) = x \lor h(i) = x
$$
and 
$$
f_b(x) \text{ is defined } \iff \exists i,j \in \N: g(i) = x \land h(j) = x \quad ,
$$
i.e., $D(f_a) = A \cup B$ and $D(f_b) = A \cap B$.
By the above definition, we get that $A\cup B$ and $A \cap B$ are recursively enumerable and thus are in $\E$ (since $\E$ is the set of all recursively enumerable subsets of \N). 
Thus $\E$ is closed under union and intersection. \qed

\bigskip
\noindent
\textbf{Solution 2b:}
We have to show the following:
$$
A \in \E \not \Rightarrow \overline{A} \in \E
$$
We show that $K \in \E$, which (\textbf{Corollary 6.0.1}) will directly imply the above statement:
$K$ is defined as $ K = \{x\mid (x,x) \in D(\varphi) \}$, where $\varphi$ is a universal function and $x \in \N$. 
Note that $K$ is recursively enumerable and that $K \subseteq \N$. 
Using G\"odelization, we can define $K' = \{y \mid y = \langle x\rangle, x \in K \}$, for which $y \in \N$ and thus $K'\subseteq \N$ holds.
Assume that the statement holds, i.e., that $\overline{K}$ is recursively enumerable. 
This is a contradiction to Corollary 6.0.1.
Thus, $K \not \in \E$ and we conclude that $\E$ is not closed under complement.



%%%%%%%%%%%%%%%%%%%%%%%%%%%%%%%%%%%%%%%%%%%%%%%%%%%%%%%%%%%%%%%%%%%%%%%%%%%%%%%%%%%%%%%%%%%%
\bigskip
\noindent
\textbf{Exercise 3:} Let $\varphi$ be a G\"odel numbering and $T = \{ x \mid D(\varphi_x) = \N\}$ (the codes of all recursive functions). Prove that neither $T$ nor $\overline{T}$ are recursively enumberable.

\noindent
\textbf{Solution:}
We can rewrite $T$ as $T = \{ x \mid \forall y: \varphi_x(y) = 1 \}$, i.e., all $x$  for which $\varphi_x(y)$ is defined for all $y \in \N$.
We define $f$ as 
\begin{align*}
 f(x,y) = \begin{cases}
           1 & \text{ if } \neg T(x,x,y) \\
           \text{undefined} &\text{ if } T(x,x,y)
          \end{cases} \quad .
\end{align*} 
Note that $T \in Pred^3(\PR)$ and thus $f \in \PR$ (we only use $T$ and definition by cases to define $f$).
Since $\varphi$ is a G\"odel numbering, we know that $\exists h \in R_1^1 \text{ s.t.\ } \forall x,y :\varphi_{h(x)}(y) = f(x,y)$ and thus
\begin{align*}
\varphi_{h(x)}(y) = \begin{cases}
           1 & \text{ if } \neg T(x,x,y) \\
           \text{undefined} &\text{ if } T(x,x,y)
          \end{cases} \quad .
\end{align*} 
We get the following equivalences:
\begin{align*}
x\in K \lra \forall y: \neg T(x,x,y) \lra \forall y: \varphi_{h(x)}(y) = 1 \lra h(x) \in T \quad , 
\end{align*}
i.e., $\overline{K} \leq_m T$. 
Since $\overline{K}$ is not recursively enumberable, neither is $T$.

%%%%%%%%%%%%%%%%%%%%%%%%%%%%%%%%%%%%%%%%%%%%%%%%%%%%%%%%%%%%%%%%%%%%%%%%%%%%%%%%%%%%%%%%%%%%
\bigskip
\noindent
\textbf{Exercise 4:} Let $\varphi$ be a G\"odel numbering and $AEQ = \{(x,y) \mid \varphi_x = \varphi_y \}$ (the equivalence problem for programs). Show that $AEQ$ is not recursively enumberable.

\noindent
\textbf{Solution:}

%%%%%%%%%%%%%%%%%%%%%%%%%%%%%%%%%%%%%%%%%%%%%%%%%%%%%%%%%%%%%%%%%%%%%%%%%%%%%%%%%%%%%%%%%%%%
\bigskip
\noindent
\textbf{Exercise 5:} Let $\varphi$ be a G\"odel numbering. Prove that the set $A:= \{x \mid (\forall y)\varphi(x,y) = \varphi(x+1,y)\}$ is not recursive.

\noindent
\textbf{Solution:} 
Via reduction from the halting problem:
We define the function $f$
\begin{align*}
 f(x,y) = \begin{cases}
           \varphi(x+1,y) & \text{ if } x \in K \\
           \text{undefined} & \text{ otherwise} 
          \end{cases} \quad .
\end{align*} 
We can also define $f$ via the basic functions, basic operators and $\varphi$ as
$$
f(x,y) = C(I^2_1; C(\varphi; S \circ I^2_1, I^2_2), \mu_0 [\lambda x,k.T(x,x,k)]\circ I^2_1).
$$
Note that since $\varphi$ is a G\"odel numbering, $\varphi \in \P^2_1$ and thus $f \in \P$.
Since $\varphi$ is a G\"odel numbering, we know that $\exists h \in \R^1_1 \text{ s.t.\ } \varphi_{h(x)}(y) = f(x,y) \;\forall x,y$; thus
$$
\varphi_{h(x)}(y) = \begin{cases}
           \varphi(x+1,y) & \text{ if } x \in K \\
           \text{undefined} & \text{ otherwise}
          \end{cases} \quad.
$$
Then,
\begin{align*}
x \in K &\lra \forall y: \varphi_{h(x)}(y) = \varphi(x+1,y) \lra \\
&\lra \varphi_{h(x)}(0) = \varphi(x+1,0) \lra h(x) \in A \quad . 
\end{align*}
Let $B = \{z \mid \forall y: \varphi(z,y) =  \varphi(z+1,y)\}$ and assume that $B$ is a recursive set.
Then, 
$$
\exists g \in \R^1_1 \text{ s.t.\ } g(z) \neq 0 \lra z \in B \lra \forall y: \varphi_z(y) = \varphi(z+1,y) \lra z \in A \quad ,
$$
and we get 
$$
x \in K \lra h(x) \in A \lra (g \circ h)(x) \neq 0 \quad .
$$
Since $g,h \in \R^1_1 \implies g\circ h \in \R^1_1$, the equivalence implies that $K$ is recursive - a contradiction, since $K$ is recursively enumberable, but not recursive (Theorem 5.1.1).

%%%%%%%%%%%%%%%%%%%%%%%%%%%%%%%%%%%%%%%%%%%%%%%%%%%%%%%%%%%%%%%%%%%%%%%%%%%%%%%%%%%%%%%%%%%%
\bigskip
\noindent
\textbf{Exercise 6:}

\noindent
\textbf{Solution:}

\end{document}
