   \documentclass [11pt]{article}
   \usepackage{latexsym}
   \usepackage{amssymb}
   \usepackage{url}
   \usepackage{amsmath}
   \usepackage{amsthm}
   \usepackage{paralist}
   \usepackage[normalem]{ulem}
   \usepackage{enumitem} 
   \usepackage{multicol}

\newtheorem{theorem}{Theorem}
\newtheorem{definition}{Definition}

\newcommand{\E}{\ensuremath{\mathcal{E}}}
\newcommand{\N}{\ensuremath{\mathbb{N}}}

  
\title{Computability Theory: Exercise sheet 2}
\author{Martin Kalany}
\date{\today}

\begin{document}
\maketitle

%%%%%%%%%%%%%%%%%%%%%%%%%%%%%%%%%%%%%%%%%%%%%%%%%%%%%%%%%%%%%%%%%%%%%%%%%%%%%%%%%%%%%%%%%%%%
\bigskip
\noindent
\textbf{Exercise 1:}

\noindent
\textbf{Solution:}

%%%%%%%%%%%%%%%%%%%%%%%%%%%%%%%%%%%%%%%%%%%%%%%%%%%%%%%%%%%%%%%%%%%%%%%%%%%%%%%%%%%%%%%%%%%%
\bigskip
\noindent
\textbf{Exercise 2:} 
We have to show that
\begin{enumerate}[label={\alph*)}] 
 \item $A, B \in \E \implies A\cup B \in \E$ 
 \item $A, B \in \E \implies A\cap B \in \E$ 
\end{enumerate}
By Theorem 6.0.2 we get 
\begin{enumerate}[label={\alph*)}] 
 \item $\exists g \in R_1^1 \text{ s.t.\ } g(\N) = A \text{ (i.e., } Range(g) = A)$
 \item $\exists h \in R_1^1 \text{ s.t.\ } h(\N) = B\text{ (i.e., } Range(h) = B)$
\end{enumerate}
We use the definition of recursively enumerable sets (\textbf{Definition 6.0.1}, for $k=1$):

\noindent
\emph{A set $M \subseteq \N$ is called recursively enumerable if there exists an $f \in \textbf{P}$ s.t.\ $D(f) = M$.}

\noindent
and define functions $f_a, f_b$ as:
\begin{multicols}{2}
\noindent
\begin{align*}
& \text{\underline{function} } f_a(x): \\
& \quad \text{\underline{begin}} \\
& \quad \quad i \leftarrow 1; \\
& \quad \quad \text{\underline{while}} (g(i) \neq x \land h(i) \neq x) \\
& \quad \quad \quad \text{ \underline{do} } i \leftarrow i +1; \\
& \quad \quad f_a \leftarrow 0; \\
& \quad \text{\underline{end}}
\end{align*}
\begin{align*}
& \text{\underline{function} } f_b(x): \\
& \quad \text{\underline{begin}} \\
& \quad \quad i \leftarrow 1; \\
& \quad \quad \text{\underline{while}} (g(i) \neq x) \text{ \underline{do} } i \leftarrow i + 1; \\
& \quad \quad j \leftarrow 1; \\
& \quad \quad \text{\underline{while}} (h(j) \neq x) \text{ \underline{do} } j \leftarrow j + 1; \\
& \quad \quad f_b \leftarrow 0; \\
& \quad \text{\underline{end}}
\end{align*}
\end{multicols}
\noindent
Since $g$ and $h$ are recursive, $f_a$ and $f_b$ are clearly partial recursive. 
Note that 
$$
f_a(x) \text{ is defined } \iff \exists i \in \N: g(i) = x \lor h(i) = x
$$
and 
$$
f_b(x) \text{ is defined } \iff \exists i,j \in \N: g(i) = x \land h(j) = x \quad ,
$$
i.e., $D(f_a) = A \cup B$ and $D(f_b) = A \cap B$.
By the above definition, we get that $A\cup B \in \E$ and $A \cap B \in \E$. 
Thus $\E$ is closed under union and intersection. \qed


%%%%%%%%%%%%%%%%%%%%%%%%%%%%%%%%%%%%%%%%%%%%%%%%%%%%%%%%%%%%%%%%%%%%%%%%%%%%%%%%%%%%%%%%%%%%
\bigskip
\noindent
\textbf{Exercise 3:}

%%%%%%%%%%%%%%%%%%%%%%%%%%%%%%%%%%%%%%%%%%%%%%%%%%%%%%%%%%%%%%%%%%%%%%%%%%%%%%%%%%%%%%%%%%%%
\bigskip
\noindent
\textbf{Exercise 4:}

%%%%%%%%%%%%%%%%%%%%%%%%%%%%%%%%%%%%%%%%%%%%%%%%%%%%%%%%%%%%%%%%%%%%%%%%%%%%%%%%%%%%%%%%%%%%
\bigskip
\noindent
\textbf{Exercise 5:}

%%%%%%%%%%%%%%%%%%%%%%%%%%%%%%%%%%%%%%%%%%%%%%%%%%%%%%%%%%%%%%%%%%%%%%%%%%%%%%%%%%%%%%%%%%%%
\bigskip
\noindent
\textbf{Exercise 6:}

\end{document}
