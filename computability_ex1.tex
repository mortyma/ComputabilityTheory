   \documentclass [11pt]{article}
   \usepackage{latexsym}
   \usepackage{amssymb}
   \usepackage{url}
   \usepackage{amsmath}
   \usepackage{amsthm}
   \usepackage{paralist}
   \usepackage[normalem]{ulem}

\newcommand{\IF}{\text{\underline{if} }}
\newcommand{\THEN}{\text{ \underline{then} }}
\newcommand{\ELSE}{\text{ \underline{else} }}

\newcommand{\ITEt}[3]{\ensuremath{\IF t_{#1} \THEN #2 \ELSE #3}}
\newcommand{\ITE}[2]{\ensuremath{\IF t \THEN #1 \ELSE #2}}
\newcommand{\ADD}[2]{\ensuremath{A_{#1} \THEN #2}}
\newcommand{\SUB}[2]{\ensuremath{S_{#1} \THEN #2}}
\newcommand{\gR}[1]{\ensuremath{R \THEN #1}}
\newcommand{\gL}[1]{\ensuremath{L \THEN #1}}
\newcommand{\dI}[1]{\ensuremath{d_I \THEN #1}}
\newcommand{\dB}[1]{\ensuremath{d_B \THEN #1}}
  
\title{Computability Theory: Exercise sheet 1}
\author{Martin Kalany}
\date{\today}

\begin{document}
\maketitle

%%%%%%%%%%%%%%%%%%%%%%%%%%%%%%%%%%%%%%%%%%%%%%%%%%%%%%%%%%%%%%%%%%%%%%%%%%%%%%%%%%%%%%%%%%%%
\bigskip
\noindent
\textbf{Exercise 1:}
For convencience, we define the macros RI, LI as in the lecture:
\begin{compactitem}
\item RI $= (i,  \{
i:\;\gR{i+1},\;
i+1:\;\ITE{i+2}{i}
\})$
\item LI $= (i,  \{
i:\;\gL{i+1},\;
i+1:\;\ITE{i+2}{i}
\})$
\end{compactitem}
We define the program as $P=(1,A)$, where $A = \{$ 
\begin{compactenum}[1:]
\item \gR{2}
\item \ITE{0}{3}
\item \dB{4}
\item RI
\item RI
\item \dI{7}
\item \gR{8}
\item \dI{9}
\item LI
\item LI
\item \gR{2}
\end{compactenum}
$\}$.


%%%%%%%%%%%%%%%%%%%%%%%%%%%%%%%%%%%%%%%%%%%%%%%%%%%%%%%%%%%%%%%%%%%%%%%%%%%%%%%%%%%%%%%%%%%%
\bigskip
\noindent
\textbf{Exercise 2:}
We define the program as $P=(1,A)$, where $A = \{$ 
\begin{compactenum}[1:]
\item \ITEt{1}{4}{2} 
\item \ADD{2}{3}
\item \SUB{1}{1}
\item \ADD{1}{5}
\item \ITE{2}{0}{6}
\item \SUB{2}{7}
\item \ITEt{3}{10}{8}
\item \ADD{3}{9}
\item \SUB{1}{7}
\item \ITEt{3}{5}{11}
\item \ADD{1}{12}
\item \ADD{1}{13}
\item \SUB{3}{10}
\end{compactenum}
$\}$.


%%%%%%%%%%%%%%%%%%%%%%%%%%%%%%%%%%%%%%%%%%%%%%%%%%%%%%%%%%%%%%%%%%%%%%%%%%%%%%%%%%%%%%%%%%%%
\bigskip
\noindent
\textbf{Exercise 3:}
$Func(P): f(x,y) = x*y$

%%%%%%%%%%%%%%%%%%%%%%%%%%%%%%%%%%%%%%%%%%%%%%%%%%%%%%%%%%%%%%%%%%%%%%%%%%%%%%%%%%%%%%%%%%%%
\bigskip
\noindent
\textbf{Exercise 4:}

%%%%%%%%%%%%%%%%%%%%%%%%%%%%%%%%%%%%%%%%%%%%%%%%%%%%%%%%%%%%%%%%%%%%%%%%%%%%%%%%%%%%%%%%%%%%
\bigskip
\noindent
\textbf{Exercise 5:}

%%%%%%%%%%%%%%%%%%%%%%%%%%%%%%%%%%%%%%%%%%%%%%%%%%%%%%%%%%%%%%%%%%%%%%%%%%%%%%%%%%%%%%%%%%%%
\bigskip
\noindent
\textbf{Exercise 6:}

%%%%%%%%%%%%%%%%%%%%%%%%%%%%%%%%%%%%%%%%%%%%%%%%%%%%%%%%%%%%%%%%%%%%%%%%%%%%%%%%%%%%%%%%%%%%
\bigskip
\noindent
\textbf{Exercise 7:}


\end{document}
